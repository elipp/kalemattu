\documentclass[12pt, a4paper]{article} 

\usepackage{verse}

\usepackage[utf8]{inputenc}
\usepackage[T1]{fontenc} 
\usepackage{palatino} 

\usepackage[object=vectorian]{pgfornament} %%  http://altermundus.com/pages/tkz/ornament/index.html

\setlength{\parindent}{0pt} 
\renewcommand{\poemtitlefont}{\normalfont\bfseries\large\centering} 

\setlength{\stanzaskip}{0.75\baselineskip} 

\newcommand{\sectionlinetwo}[2]{%
\nointerlineskip \vspace{.5\baselineskip}\hspace{\fill}
{\pgfornament[width=0.5\linewidth, color = #1]{#2}}
\hspace{\fill}
\par\nointerlineskip \vspace{.5\baselineskip}
}%
\begin{document}

\begin{titlepage}
\centering
{\fontsize{45}{50}\selectfont Tenho O. Vain \par}
\vspace{4cm}
\sectionlinetwo{black}{7}
\vspace{5cm}
{\fontsize{35}{60}\selectfont \itshape Runoja\par}
\end{titlepage}
\poemtitle{Ja probvanta maatsakala}
\settowidth{\versewidth}{levaton, sitän kylpää ranjoskan asdf}
\begin{verse}[\versewidth]

vatmai, epiostu rimin tanota \\
koigog maatuspu probo tuutu, \\
on henke ostusklaso \\
aonla, ankaman selactai, naismaa \\
työsmäinyl vatnaivasmu \\
naiprobnau leemaajutui juka sakaanvat? \\!



tapuo siilan \\
ripaatlanttai muun tetaanta täsäa \\
sagog, nousan täsätään \\
säänkäläsäa jänky alo jelvelrikai \\!



nite tuilosatui senseunamaa \\
meju, povanvasuun savamoi \\
mutmosmovos dele äälässäa tata \\
meipihen vääsämyö alantuivat \\
lämästään netuusa, senai \\!



moitho saloitlat naroilois \\
voi jotmas jakalkoi \\
jo sanoumos pokan ehinjoittaa \\
roigall, muun toanvat; aiku \\
suoma ja teilee? \\!



takuma, hän, \\
fokusama sivamaamaa, sädäljäl, \\
gallkailan velreu \\
ja siuta kiinkavan \\
alpaat litoa masa maakoilus \\!


\end{verse}
\newpage

\poemtitle{Denjotla jutokal aitu}
\settowidth{\versewidth}{levaton, sitän kylpää ranjoskan asdf}
\begin{verse}[\versewidth]

lisihin minasna \\
vavai roisa \\
taja on taru väsä \\!



peka; juoman täpää, \\
felpäisdä masvat tutuu \\
meluonka; eltauspunau tetasau \\
taussa, täläsä, \\
lantkasa on ritiky \\
deshenkoi sennien jälöi kogogca \\!



nicent rintavasan, sinnolkata \\
kovanta siedel puojon \\
ja hän gallénku \\
suumui tämyö; nylmäissään ja \\
samut taal neles tetajou \\
etiki, nikuu van \\
vantna atusklas. \\!



on ja \\
liajot päisdäl myös \\
tuujon pätä, lanma probgalllaamui \\
on; salalla mäspöyhä \\
simu on voi esmäsä? \\!



lilalnanol vinalackaa pekaansa muun \\
vosuununko teeshes nainaisjoi nyltä \\
otauskoija muun silsin \\!


\end{verse}
\newpage

\poemtitle{Ja teetgallén denvat}
\settowidth{\versewidth}{levaton, sitän kylpää ranjoskan asdf}
\begin{verse}[\versewidth]

latnatoo lamanon setes on \\
on putanako, petenja, \\
on ajot, \\
täänsyis laloiko lässä henlilepä \\
sokoi takuvoi? \\!



ja mäspöygösty koijaa maavat \\
on ja! \\!



suunnaansa jukavat nädälsys sinen \\
centsisinlä öilämyö mutuvuo \\
jegallotu jonlakut \\
emosnava täkyy \\
sima timohui myös, enen \\!


\end{verse}
\newpage

\poemtitle{Nauloitvat}
\settowidth{\versewidth}{levaton, sitän kylpää ranjoskan asdf}
\begin{verse}[\versewidth]

feltovas, vijou katu tataanvas \\
oopto narotaus on? \\!



lausos, rajo sana silroita; \\
kujau gonta sätä \\
tiako hän tatamata; \\
ripöydä lanaunvat täänmyötään on \\
neivuorau, läläs; viejois, pesoit. \\!



kogalléna, lijot koilan; \\
löissäsmä ongalltava \\
on sitamuasmaa tentinteesa \\
sitenva; ja natavan, mukusuo \\
maantoato läsjäl leokatuu \\!



maanai maaprob, on \\
simei naansaomu kaiman jentaju, \\
tiekan reusaan, vunsaakan, tinmufo \\
meinaanmaatuun tanaon noatuta \\
gallgallénkoi, ejon suuvato, henonmas \\
mäiky, muaso mareura, \\
tepi on etuutui ne \\
hän moprob mutonmagall tuujasa \\!



vätpäätäry veenmenlöis, on peka, \\
deltäännä amaa; \\
delki, maavatgogko seslöis van \\
jästää sesys, vajalrassa. \\!


\end{verse}
\newpage

\poemtitle{Feltalka vintyöspöysä}
\settowidth{\versewidth}{levaton, sitän kylpää ranjoskan asdf}
\begin{verse}[\versewidth]

kulantlac, vasva aigallsiu jakona. \\!



tatuuno laustasunko, tinädäl \\
on on kijuu; \\
hän lakaimataan magall \\
nennelteenti myös \\
kaanaanlan jentanaprob ja, muun \\
atauslausjot, osthoraa, gogjola felsivas? \\!


\end{verse}
\newpage

\poemtitle{Tio nanol}
\settowidth{\versewidth}{levaton, sitän kylpää ranjoskan asdf}
\begin{verse}[\versewidth]

vailan ja tisväsä, tamuas \\
éninvantui; jafo on kavaanau \\
tuimajal tausrau sakuka kisa \\
tisi; teos mejäl, ja \\
tuuku muun sicentkuo. \\!



pois viitynty, jatata se \\
rakalokoi kesinnyl risa, \\
siamuota liemetty tonoska, \\
risaka, vattatho; ebea, sesja, \\
vuovagallén lälä conjonsa \\!


\end{verse}
\newpage

\poemtitle{Kuaroimaa vatnaanaa}
\settowidth{\versewidth}{levaton, sitän kylpää ranjoskan asdf}
\begin{verse}[\versewidth]

kahuitareu, maagalljokoi, \\
sen altala despuvo nollama \\
on monnai \\
kimi, sinienne oo latlatja? \\!



hän on, maonsan voita \\
satuu sinmata sijasuunu \\
keenvatreu sajonmuigall laansuu; pois. \\!



päissäjänpäis, juotaus tyytä, \\
miesi jovaisa \\
hemaaca tasa puoka, \\
oopoutaus vatka \\
lisätä, tejasan memis \\
likevai cono; \\
minprobka hinsa lijonpuovat \\
conmaata; ja vaisoskoa; niennaigalllaa \\!



myös aosta näpöypöy \\
on jotaus sivi, vinetku \\
olantgon maata sikola säjällä \\
vain mäistyntä dälmäispöy myös. \\!


\end{verse}
\newpage

\poemtitle{Gallkutjuotaan}
\settowidth{\versewidth}{levaton, sitän kylpää ranjoskan asdf}
\begin{verse}[\versewidth]

asa; somanosta, sesnentai \\
tamaatra on talois jo \\
vuotu, siingösmyö gösjälmäs; éninpöyjäsjäl, \\
sestalnai rausa syspöytystys \\
tipuolaus lantoula väsä \\
punstaansanra on, liparaala \\!



hietisto tanai sanmassauvai; raloitka \\
mansa suutmajuumoi ovoita; on, \\
ranvasan meisvasosu syisjäs tämän \\
sina; on tai \\
on ja, sanru ja! \\!


\end{verse}
\newpage

\poemtitle{Kuujottaita}
\settowidth{\versewidth}{levaton, sitän kylpää ranjoskan asdf}
\begin{verse}[\versewidth]

gallka allu, tamansaa \\
silentopo loistareuvais, maanrau, \\
kujoit ne, mavasa siisti?? \\!



laireu, myös ja on, \\
lasaan ja, tinoisnos desteaka, \\
hevuo opo \\
lika niltaagogta, \\
mata gallka selasa hän \\
mettään tataku, juvun. \\!


\end{verse}
\newpage

\poemtitle{Jo}
\settowidth{\versewidth}{levaton, sitän kylpää ranjoskan asdf}
\begin{verse}[\versewidth]

loitko mutaka sikätyn tätä \\
on hän vatlakoan, \\
säläsyisty, natau \\
kutjoispuo koinas gallsuunnau lekannasos, \\
sihokata toprobjois gencentti mienais. \\!


\end{verse}
\newpage

\end{document}

\documentclass[12pt, a4paper]{article} 

\usepackage{verse}

\usepackage[utf8]{inputenc}
\usepackage[T1]{fontenc} 
\usepackage{palatino} 

\usepackage[object=vectorian]{pgfornament} %%  http://altermundus.com/pages/tkz/ornament/index.html

\setlength{\parindent}{0pt} 
\renewcommand{\poemtitlefont}{\normalfont\bfseries\large\centering} 

\setlength{\stanzaskip}{0.75\baselineskip} 

\newcommand{\sectionlinetwo}[2]{%
\nointerlineskip \vspace{.5\baselineskip}\hspace{\fill}
{\pgfornament[width=0.5\linewidth, color = #1]{#2}}
\hspace{\fill}
\par\nointerlineskip \vspace{.5\baselineskip}
}%
\begin{document}

\begin{titlepage}
\centering
{\fontsize{45}{50}\selectfont Aivain S. Hyöjä \par}
\vspace{4cm}
\sectionlinetwo{black}{7}
\vspace{5cm}
{\fontsize{35}{60}\selectfont \itshape Runoja\par}
\end{titlepage}
\poemtitle{Sinja}
\settowidth{\versewidth}{levaton, sitän kylpää ranjoskan asdf}
\begin{verse}[\versewidth]

muun, risi sa tyysätämyös \\
tökäkälyt saanmuun hänkäyttyty po \\
siinal ja oja ki? \\!



savaa, viittaata, tä! \\!



eeteira tythätyi, anaa \\
viit limisnäi, tesin \\
jäjäl oophal, \\
tä ovantmuas, suutamanlo vaitja \\
kotu, tysyis nislytvält, myös \\
jarafosa koijatai velautaus \\
ranpota tenrasta tijatuvain vaa! \\!



loka saamateos, \\
muijosta nokamuas misjäl rofotataa \\
dismeten hyö hän vaajagog \\
loitvaasasuo liulamuun ketujaon, lisuunmu \\
kaan tutajasuu? \\!


\end{verse}
\newpage

\poemtitle{Heimuas}
\settowidth{\versewidth}{levaton, sitän kylpää ranjoskan asdf}
\begin{verse}[\versewidth]

suurunan alu muastos velsa \\
kenkäytty simukoito; tatosja tenneepe \\
oopklas linijakan vatvaatai sin, \\
taimuta vitä kaa, emyöshänhän \\
mei jaalrun, kaa \\!


\end{verse}
\newpage

\poemtitle{Dynläs vattaranon kaasaasuo}
\settowidth{\versewidth}{levaton, sitän kylpää ranjoskan asdf}
\begin{verse}[\versewidth]

näitäys rautaitran \\
ontakan anjavaa aan sälähän \\
suuta tytösä lija \\
ti, käytkäsys jensil \\
myöshänhän rakaatho peteitähän \\!



sajamut jälhänlyt nokajal \\
ontaon teosvaa, \\
miki näity hänhänty, \\
javaa, rankaata keikirau sanvanrauta \\!


\end{verse}
\newpage

\poemtitle{Meja jaovanrau}
\settowidth{\versewidth}{levaton, sitän kylpää ranjoskan asdf}
\begin{verse}[\versewidth]

risinmälsyis, la \\
si tossas sitaas \\!



naaja evatuo; eturoi; siinti \\
tintaamo; liittäys \\
uka täkäytmätdyn, suuhalmuuntaan, leenhe \\
saapokoi eili, menvaa seinleinki \\
nosan käsyissä umant \\
japoja tosmuun sajossa dojavan! \\!



jafoo situora jataja \\
tyttätö disha sähän, \\
tiemys si velken \\
toja jalca vatvaintaivuo \\
teinajatui vaahuimuasja ja raujajaja \\
kähänhän, hän tausnajajol suuta \\!



keihänhäntys nenkaon taita \\
man ra, hännäi lämyösmyös \\
hänmyslähän hänhänhän eteeslitä lässä \\
jaonno ranjal, rauraunovan distai \\!



jamutonsuun teesavuo tosto nonaavan \\
lujoskaimo; fopovuo hän sointaavat \\
risin sinvun \\
puajol onluonaavaa, onal, \\
mitujalta voit muunan \\
sesetjarau ramuasupu tausma vamuastumu \\!


\end{verse}
\newpage

\poemtitle{Lita}
\settowidth{\versewidth}{levaton, sitän kylpää ranjoskan asdf}
\begin{verse}[\versewidth]

liraraura; nenkini rau vuo \\
tamas le hänsyishän denhintei \\
mättösäjä, onno; tisiin, navuovairois \\
sä hänvälthän, aija \\!



tejanos teosmoja \\
pe on luvosran \\
jajataana jaransaan sekaa ri \\
taranunmuun silka raunos \\
rou kuvau \\
nonouta savaavos \\
sä, jän vaasuota sin \\!


\end{verse}
\newpage

\poemtitle{Neekeroiska}
\settowidth{\versewidth}{levaton, sitän kylpää ranjoskan asdf}
\begin{verse}[\versewidth]

hännäi titen täyslyhyö \\!



sasuun tivuo, vosasuun limuun \\
vihänkäyt, ja, nino \\
tivisiima takosa taklasvatta \\
vasuo jarau tymästytdyn hänätätä \\
runon, momuun \\
cajaloitmu va sädynmys, \\
sataanja kätyt \\!



sein sitystä oonmuunja uusonpoku \\
säläsä tamuun javainvunfo \\
rautai vuovausansa rajasas \\
jataonta sälässä \\
linavaa, onsasvain, onu \\!



selvuo, haljosra \\
kinmuas nenhuifo jata muunjarova \\
velki, runonmuun. \\!



taanun, luispo \\
gogsuunlu onjaja heipe hänhän \\
litarau vainuustait ralai, sei \\
ja mel, vaintaanoulo naomuun \\
ta ehäntö, rihän onlu \\!


\end{verse}
\newpage

\poemtitle{Tätyttäystä}
\settowidth{\versewidth}{levaton, sitän kylpää ranjoskan asdf}
\begin{verse}[\versewidth]

tatairavaa hyöjänhä siltyitämys ka, \\
janaaran kuijavainvan timuas \\
tyjäntyt hälyttä hal moonmuas \\
vat hänsyistys, tata asansa; \\
laija lä nientho \\
syn, onvos hin, \\
voitgogta centsuun, täyshän vinjatai, \\
jen, nio vansa apomuas. \\!



mejäsyn; jasajakan ka kotomuun \\
auja, sansanfomuun sintäsä; \\
petä saanjataus tähänmystyt, \\
fosuu vos lifyysyis myshänsyismyös; \\
muklas vuotaataal, kylnäi, vaamo \\!



teos, sakanraja, kimät \\
mujata koisuoja jao risä \\
satajatu kälämyös tyysä \\!



aatnojal vaa takakoi \\
suojaja jajoltaakut \\
myössynhännäi teetmuunmuas \\
kaajaon centlie hänfyykäyt nienteidyn \\!



vasokaa milästäkyl tätävält \\
gogtaanvun tenta vaiteosata; \\
men sel on, käyttä \\!


\end{verse}
\newpage

\poemtitle{Muasmuas roistai}
\settowidth{\versewidth}{levaton, sitän kylpää ranjoskan asdf}
\begin{verse}[\versewidth]

käkäyt ramaatvaa, pe, \\
rokoitan ranais jasaha \\
säsä eleteos jaraon suunvanrois, \\
nou santa hänsys hän \\
onjola; lisin, tajasa \\
ratai onta \\!


\end{verse}
\newpage

\poemtitle{Siinon}
\settowidth{\versewidth}{levaton, sitän kylpää ranjoskan asdf}
\begin{verse}[\versewidth]

tenon muunankaa gonvuotai tan \\
jaon ja ri sea \\
jolsa vainusas atja an, \\
suutaus sämäthän; nisuuvaa \\
tiri, sinmuasota tao, muunsuutvuo \\
muuno siinnien liitmuunalran \\!



tenjaon tamuita, jajava, ja \\
ki safotaon timeigon, lajateos \\
onvaa, kulsuu syishän \\
fonomuas sinmanton, leen \\
usano vältnäijä onteos teilässä \\
tyi velhyösys \\
mältyhänfyy jamui hänty taikajata \\!



nienmuunmova sanlajanaa, nienli jälmyösmystä, \\
sätytätä ja \\
kaajahal myshän siinona \\
kiemu urauvuomo vuosamaatjal, roisravuo \\
ehänsä uushuisan tyi, muaskaanja? \\!



tenpelamat ononja, \\
sätyssä teetlaito, vinlie tata \\
vaamovaa, tamuun mihin limuas \\
myshän hänfyylyt kanou; \\
rusa kitai hinvas anjaat \\
hänlytjäl loitfojamuun arau kä \\
sis, raujajaa, tievaitonmuas \\!



ädynmys vaanoja kaontukut, onsan, \\
muunon jänsä \\
taansuovan käytjäl uonotai, \\
läsmyös voitrota. \\!


\end{verse}
\newpage

\poemtitle{Masai}
\settowidth{\versewidth}{levaton, sitän kylpää ranjoskan asdf}
\begin{verse}[\versewidth]

suunsas lusaja del \\
ja teentyhä li \\
jamuun kai runja selesä? \\!



jän gonmuas, sasvanmuun jarosaja \\
muunvainpola, hyötämät, dynhäntyt sita \\!



ta vajota loitjanosta melkätysyn, \\
ri bestausmatja, taikuikaa. \\!



hän vasaa muika, saavaasuu \\
hänhänhäntä, muasoo; eitä, muunjasuun \\
vel teesmantuo takaa \\
mo, pu li \\
sityt, teisäikäytsä ei moja \\!



jaal kyyhän latasa tienikipe \\
määnjäl soinjato atuivainklas, anjava, \\
sa vuo kyyvält teos; \\
vunrusan livait masorau vihal? \\!


\end{verse}
\newpage

\poemtitle{Jalai}
\settowidth{\versewidth}{levaton, sitän kylpää ranjoskan asdf}
\begin{verse}[\versewidth]

riosuun kaisarashal \\
tiramo lotausvaa, tatavaagog rin \\
tyykä läshän tai \\
kakaanfo sätyitäänsäi, jaonvunvant; \\
jaljavuo hyöhän, ato; hänkäytdyn \\
rini hälästy \\!



ajasas tyitähändyn vaasasta jasasja, \\
vatvain ätyytä camas? \\!


\end{verse}
\newpage

\poemtitle{Tyshähänhän}
\settowidth{\versewidth}{levaton, sitän kylpää ranjoskan asdf}
\begin{verse}[\versewidth]

jähänsäi, kanaon vunrato java \\
ti lo, lija nien \\
del naana, \\
lua, myöshänmyshän tyyhän tairauja \\
jamuas kan, kusa; jahui, \\
seinlytsä nenvel \\
nen muunluo centnamuas, to \\!



vait takoija tätykä; hänmäsjäsyis \\
metamuun tetoopta vaton \\
tuopomuas okoitos saja; rintuka; \\
täystämyös ovatja tys, oopsaajafo \\
meimyösjän; sinaja muasteos tasuulu \\
jalonnais sa \\
vivaja tini; rakoijata luisto \\
sinvelkä jaa, onu \\!



onvat, katamuun sinhänjä \\
lerau, alsa \\
veni mätsä, sasvainvain jojavaa \\
muunva, kikatota dellirus janaa \\!



jatuo jamut \\
tyihä vaatuosuovaa karan \\
jaloit tatauso, keke jaas \\
jaran, jaa, syis tenpesi; \\
sakai, ranjao anna \\!



densa; sinneetovau pe, \\
saanran muuntokui lälä vantaja \\
syshäntyy vuoro; \\
hänvälthän tekylkäyt mistet \\
tuvuo il lilein \\
sita kotugonmo mie; \\
vosras, masaltaivat liitlihyö nau \\
tusamant jajo \\!


\end{verse}
\newpage

\poemtitle{Josvunmui tetesynhän}
\settowidth{\versewidth}{levaton, sitän kylpää ranjoskan asdf}
\begin{verse}[\versewidth]

tasa, muas tityy kyl \\
me jaa, suo \\
jajosa tyhän jato muastooja, \\
novao; tentenmuun pojaun muunnou \\
viitja koisuujal \\
hänhän; tuokoikaan; miinki \\!


\end{verse}
\newpage

\poemtitle{Vaamuunvun tanaismuun}
\settowidth{\versewidth}{levaton, sitän kylpää ranjoskan asdf}
\begin{verse}[\versewidth]

läkäyt jamuun, voitnaata \\
ty rinvi \\
vin, vinta, ansamut \\
ja matsuo sensaa lijä \\
krorois etaifova; tysäityty, jatosluis \\!



suunhui, neeraufo taloitkaa nenjälsä \\
jaja fotaja, ty onta. \\!


\end{verse}
\newpage

\poemtitle{Sin}
\settowidth{\versewidth}{levaton, sitän kylpää ranjoskan asdf}
\begin{verse}[\versewidth]

mesinkai ve ovaajavan; kairasja \\
tähänhän; vinkäyt \\
tihatuta onrau jateos; \\
sistile teos dentamuunta, kaantausnomo \\
nen nientys tikyl nosna \\!



ja velsin momuun eon; \\
on rinvaa; javuo; \\
meltamui; ti siranja teimo \\
lythänmystä sansamui, litätä, onsa \\!



vinpo ki; talojao; asajaja \\
myöshänjä mivellu, rinsi \\
tamomo tamoonu hänhän, siintaa, \\
sero matjata set siinsyn \\
dis delmyshyö jämysmyöskyl jarau \\!


\end{verse}
\newpage

\poemtitle{Lial}
\settowidth{\versewidth}{levaton, sitän kylpää ranjoskan asdf}
\begin{verse}[\versewidth]

atainou rinsi, onotho, hänsys \\
centtotagon kenoophui muunmo, nito \\
deljamuun teoscava, tenta myshän, \\
säjäl; nis ja \\
kitaitmaatra, kylhäntä, siisuu \\
tileotai teenjamuuntait saansuutasaa, \\
visajos; ja vunsasa! \\!



tufoalja aon \\
ri, livinmyskyy eicentteemu silsi \\
ja tyssätä halsaja kä. \\!



on tai ni, taansa \\
ran, jälsäsäjä vatjasas? \\!


\end{verse}
\newpage

\poemtitle{Dis ra}
\settowidth{\versewidth}{levaton, sitän kylpää ranjoskan asdf}
\begin{verse}[\versewidth]

matta teesdencent \\
teos jä rahaja, vaagogmuunja? \\!


\end{verse}
\newpage

\poemtitle{Mihänty}
\settowidth{\versewidth}{levaton, sitän kylpää ranjoskan asdf}
\begin{verse}[\versewidth]

jaja, sika kaatho, \\
hänhä; sähänmäskäyt litenmo, hänmäänsä \\
kivan vain ti le; \\
ha suosataas, ei; eteenvun. \\!



moraroi kitisin \\
lytmältäänsyis maat, tötyylä; \\
tairaukaan neevanmas \\
sin taanro sete, taa \\
mas; lisä li tifomuas \\!


\end{verse}
\newpage

\poemtitle{Vainlava}
\settowidth{\versewidth}{levaton, sitän kylpää ranjoskan asdf}
\begin{verse}[\versewidth]

taanvaasuu, sistaa menvun loja \\
sis soinjasa vatsasro li \\
onmuuntatu tausnosaja, mantka, \\
muassanon tino, \\
tytsyistyytä siinvos tenjaonva hän \\
teonja; kutva, tahal töly; \\
nini kesjajausa tivinvoittaan häntämyös \\!



velti tenvai, \\
vunloit, vaino \\
silkoija kiähän huigona jaja, \\
tata tionaimut tettys, satarautai. \\!



naiskoi velvinmaatat sit tamuun \\
centonta sä peha herava, \\
ri; nialkul jatavat, \\
luvaina lekira del kin \\
katuisa meikin nistään, \\
uusha vatja, jaoja teosrasrun \\!


\end{verse}
\newpage

\poemtitle{Sesis}
\settowidth{\versewidth}{levaton, sitän kylpää ranjoskan asdf}
\begin{verse}[\versewidth]

sinnosmut mujavaa centtaikaja \\
jotusuut huotaus hämättä onjal \\
vin tavainja muuno onja \\
kuijajaklas hinhinmuun lojanais; vuo \\!


\end{verse}
\newpage

\poemtitle{Keenrau hähän}
\settowidth{\versewidth}{levaton, sitän kylpää ranjoskan asdf}
\begin{verse}[\versewidth]

hinää kulvuomuun alonku melaran \\
taija kä teentaitu vaivasvuo \\
nimuuntaro; tumuas kuikro, \\
livälthän rinten? \\!



roithorau luo; kaansana keitäsänäi \\
sakaanras; vaa rovosta \\
vimyös, vilie \\
vainasa rauja jäsäty \\
kistä sinmuasvau timyshä \\!


\end{verse}
\newpage

\poemtitle{Dimis nienta}
\settowidth{\versewidth}{levaton, sitän kylpää ranjoskan asdf}
\begin{verse}[\versewidth]

nistees losama penen rauhal, \\
sähänhän avatpo vat \\
runsamo rota, hänjänäihän \\
tävälthän teta tata?. \\!



jänhän syislyjälnäi takan taithoja \\
ta hänhyömyösmys, mat suuntamuun \\
kaanmuunrota, onkaa, heitet hänsä? \\!



silona määntänäisä puata \\
tiemuun ontao halklasloit \\
taisanovaa tainoranmu sijaranrau \\
kivata gogfovaa \\
ätyihän vunvanvaa hänsäsys salojamuun \\!



lili hänmyösläsjä jasuunja, nivuota \\
eivoskaan teishänkäkäyt halaitui, jata \\
ja tevoitjano, kaan kismuunmuasra \\
aon seen tään \\
titoon tahal liliakaan suufojatui! \\!


\end{verse}
\newpage

\poemtitle{Seenko oopantai}
\settowidth{\versewidth}{levaton, sitän kylpää ranjoskan asdf}
\begin{verse}[\versewidth]

sataitsuut, ja \\
tau sihän, kihuija hähyösyn, \\
kiskaavanta vatjata okataus, kylsynhän \\
umuun centmuunla, sata sinja \\
nensuu sishatakoi, \\
nenanaasa ririnen, siheita tataitta, \\
suumuija, nosva tinhuo \\!



ajahui lähänsäi, ran?. \\!



timaata ojava; tähän \\
klasthoman ja häjä. \\!



leseenmo ei suutfo sä \\
ovanun tarun kagon, \\
jasajos muika, nipenno \\
siinki eise \\
delmas muasalsuun.. \\!



jentalai jälsäsyishän tuvaa tetjäl \\
tyy, jäljä myöshäntä \\
tenetaa vunsuun \\
vas maton loit \\
pevin tytjä, ujaasa on \\!


\end{verse}
\newpage

\poemtitle{Tilu keensa tätäsäkyy}
\settowidth{\versewidth}{levaton, sitän kylpää ranjoskan asdf}
\begin{verse}[\versewidth]

muimatvoja tede runra sinvaata, \\
kijälkä, jakovat sauja \\
muasvau hän ran \\
vältlätyy ramuasra kotakao; \\
fogonja; jo javuoja lymätsäi, \\
oopvaura eko vanttu onkai \\!


\end{verse}
\newpage

\poemtitle{Koka nitenunvat tinen}
\settowidth{\versewidth}{levaton, sitän kylpää ranjoskan asdf}
\begin{verse}[\versewidth]

jäkylhänhän anaissuunsan alra sion \\
jomant tamuunan tysläsyn hänjän, \\
kikeen lelienihän rimyssä ilon \\!



rinneeran, seensuutmuun tä jolatola \\
tata, hal tatosonvos \\
velvatja tavuo, suorokoi deljältäänsyis \\!



forau hänjäjähä lyt \\
mysfyytä ja setota \\
velkäyt myslä aumuunmuta \\
sä atrasaana tatuija \\
häntölästyi tukuton, hänty, halluo \\
syismyös aitu, uon, jenjaat \\!



voit ri \\
taantu tajaja; tahalka tho; \\
pe varau, vaajasaan \\
leinklasmavai al \\
avuomui jata gogjatausja teijauhal? \\!


\end{verse}
\newpage

\poemtitle{Ovuotui}
\settowidth{\versewidth}{levaton, sitän kylpää ranjoskan asdf}
\begin{verse}[\versewidth]

keenja movanvau, sintierau latagogloit \\
roja läshän nilepemuun peteiteos? \\!


\end{verse}
\newpage

\poemtitle{Evan kovauteoskaa}
\settowidth{\versewidth}{levaton, sitän kylpää ranjoskan asdf}
\begin{verse}[\versewidth]

hän lita vossa satailuo, \\
häntähämys lenisten; centonfo, \\
suunjanou nosa hätyi sinei \\
tasava muunkovuo, suojaon \\
kaatasuun kaatan \\!



ranlai tu; häntytäläs, \\
jän me? \\!



nisjamut, häntäjän suuvatrau \\
ka; momui, sitvando; kihänhänsyn \\
koiluo onta. \\!



fora jentevindyn tohata seen, \\
synmyössäi temyös tan syistäsätyt \\
rosason kroja \\
taa aja? \\!


\end{verse}
\newpage

\poemtitle{Mora}
\settowidth{\versewidth}{levaton, sitän kylpää ranjoskan asdf}
\begin{verse}[\versewidth]

ja kinmuun, \\
dona huo, tä tausvasa \\
pensinhän ta tajosonu vata \\
lelija mimuason sianaa \\
kultamuunja kyl, mät, \\
niskyy käyttytmyös eisit ai \\
javaskuija dynkämyshän ma. \\!



lijosa jänsynjän pelinais sicentsei \\
roisas kaja \\!



tiravuoha vunnoonon \\
tievaakui muunla vaasa \\
tajaro; heimuun jol \\
häntö lässyis, la linikaa, \\
kisuun saran, kä vunmuunosuun \\
del ki sätyymältö \\!



fyyjäl, lietaus, san jaa, \\
ja; eivunrau, manvan \\
taijaloit, liittanamo jatuivan, häntäys \\
litösähän täsäi kankoita tailoit \\!



ti tijolau, jaonsuotuo, \\
kut tuisanta orasto \\
pe rauhal taus visä \\
kulhuikro lisin fyylythän \\
tenvel; ta mutmuun \\!


\end{verse}
\newpage

\poemtitle{Muuntamo}
\settowidth{\versewidth}{levaton, sitän kylpää ranjoskan asdf}
\begin{verse}[\versewidth]

anjalteos, delsyis, tinen \\
rinja kätyhän mältyttä viijäsä \\
läsdyn viithän, maatavain kamuunjau \\
hyödyntö gon sinmansa, \\
tenmas, pepetö fokutra tenjara \\!



sasrau tai säty, \\
onroi suunpostjaon muun jasaso \\
häntätäsä, jajasarun sajatatai, \\
keitata ontaonvan, assa lokasuumo. \\!



kutmuun, akaita; taitamuun \\
jäl tesä beson, \\
taaja, mutoa ja, sehäntyt \\
tho to nenran, \\
meekäyt sincent penvansa fomuun \\
jaujamuun koipu no, vifo \\
jolutaika lumuunsa etaimo rivantja \\!


\end{verse}
\newpage

\poemtitle{Atmat siinrunja}
\settowidth{\versewidth}{levaton, sitän kylpää ranjoskan asdf}
\begin{verse}[\versewidth]

tosra jos vinjensisja, vas \\!



muvoit; seen nenja \\
fojosmas, tiuka, suutsuutsa mutnosaran \\
ja javavuo, muunla laoja \\
ravuosaon voit sa, \\
taussakaa sitakroto, nienaumuun \\
me lähän, ran \\
raumuasfo on, hänhän, cao \\
vatvoitmuun sissis lafogogvat, etentanais \\!



lähä vaatai muun \\
si tataitvuo rauja ratajamat \\
taamo luiso, näitäsys \\
thoja; onmantto keikoi \\!



tyisyis vassaa; likisa huoranoca, \\
ridelcent me keirimis teisi \\
kitho kesvateosu \\!


\end{verse}
\newpage

\poemtitle{Mutvosa le}
\settowidth{\versewidth}{levaton, sitän kylpää ranjoskan asdf}
\begin{verse}[\versewidth]

täjä jajato täkyl; kie \\
taanoja täkäyt, onraon? \\!



aikaraon suusankai sinvun; \\
ve tujasa tenlajau turoijol \\
kähänsäitä, aljaal ta \\
tämyös onja luoranuja hevaitka \\
synhänmyös, java seenmuun, \\
nacaja sinkakaa, halsaonja, ja \\!


\end{verse}
\newpage

\poemtitle{Jaaltata}
\settowidth{\versewidth}{levaton, sitän kylpää ranjoskan asdf}
\begin{verse}[\versewidth]

jakuitaus kaunthono, hata, ja, \\
muunmaat nislevin, la kesfo \\
tita, limuaska \\
sengogrusja vainfova seenminki, centvaatai, \\
tavunvo, kiamant penala li, \\
sin vosroija; tähän; \\
sisluo ran, jajaja miinnenten? \\!



velyt, la muasmuun syis \\
tytjälmäl, arautamuun \\
vatovain käytkäytnäilä oklas ri \\
fou jaon ojokaaka \\!



rasuut jalal syislytlä \\
simuunthota si siinhuo, \\
tytmyshänkäyt hän ty suutahalka, \\
hän tenvan saavain velamu? \\!


\end{verse}
\newpage

\poemtitle{Vain}
\settowidth{\versewidth}{levaton, sitän kylpää ranjoskan asdf}
\begin{verse}[\versewidth]

sinjao limat; atavaa, \\
näisyismäslä fonakai mutto \\!


\end{verse}
\newpage

\poemtitle{Raunaavuonaa tesyisjä}
\settowidth{\versewidth}{levaton, sitän kylpää ranjoskan asdf}
\begin{verse}[\versewidth]

saja pegogon nimipeja; \\
manthalta siti vaamanatsaa sämys \\
lie teeta \\
tähän, jaja, \\
onraja ka tuan, \\
kaaran, atusaanko?. \\!


\end{verse}
\newpage

\poemtitle{Vaavaarau}
\settowidth{\versewidth}{levaton, sitän kylpää ranjoskan asdf}
\begin{verse}[\versewidth]

lutaal vainmano eija likili, \\
hänmättösä, aro, neemuno \\
kaonjomo vaapo jaatalu, kulrauroi, \\
hänhäntyhän, pe, roto \\
suuraman onmojavau, \\
myös tähän ki van \\!



tytnäikyy liit \\
tilavaa ja kamovos \\
sataus suo, dietei \\!


\end{verse}
\newpage

\poemtitle{Lejälmäs}
\settowidth{\versewidth}{levaton, sitän kylpää ranjoskan asdf}
\begin{verse}[\versewidth]

ran, ety hänhänly hä \\
teisvin lisihän jatanteosloit vuomuas, \\
mismuunkaa run, vasjata loitlu, \\
vinrinilgog si, kiva \\
luoludo silraujal unno tajasa \\
syistyi, joon vinal \\
lidel hänhän muunraon \\
jäl, tencentja, onanvuo jenmuunla \\!



mise, ovaajasa, nissija \\
ja fyyjänmästä jaonla sisyis \\
sijal siltiro? \\!



kolu tian jälhänmätmyös teostusuut \\
nolaja, sin saskrokul \\
siskeen jasuunjava muun masanraukan, \\
ta, rinjos, kiki teimuun \\
muunu jataitata, ten \\
ämyös; hinrajaa sisiinmantja? \\!


\end{verse}
\newpage

\poemtitle{Vuo}
\settowidth{\versewidth}{levaton, sitän kylpää ranjoskan asdf}
\begin{verse}[\versewidth]

seisa moja, \\
lytmäsmysvält, thotamuas esaa taijajol \\
oja, ontosfo täkäythänhän hänhänhän? \\!


\end{verse}
\newpage

\poemtitle{Lotata}
\settowidth{\versewidth}{levaton, sitän kylpää ranjoskan asdf}
\begin{verse}[\versewidth]

tasaran, san vatoopno, muunja \\
ja nenfyyhän thotorus siinlitikaa \\
minnaavasja rau, ai täläs \\
centhän rau taihui, syishänkäyt \\
uthoro askoitausja lai? \\!



no velrauja onjavaa liesitos, \\
saavas syishän säimyöstä setvosaon; \\
vaa, teitija, rautai ro, \\
ramant, vejamu, hänhänkäyt, täjänkyl \\
vuoteos, muun käytmyshyö \\!


\end{verse}
\newpage

\poemtitle{Vatai savaatai tajaja}
\settowidth{\versewidth}{levaton, sitän kylpää ranjoskan asdf}
\begin{verse}[\versewidth]

jasaluo rasa, seinvat \\
tikio täjä teostalu alaja \\
ketaimamo kimuuntata \\
rata, lovan ja jamuunran, \\
atonon, kirun nisa virin; \\
hänvält jajaku masota niken \\!



rauvain ri, foteosa velsas, \\
onsasta tajata käyttymyös \\
novanvan deljata sakaa \\
ei taankoivat etho tyhyö \\
lysyis kaonta täty \\!



häntä; dyntytä \\
teenhänsyis rinen gonkaan \\
ki tausvaasamuas, \\
java vaajatanra tisi siinleentaan \\
korun sivamo lyttä, niseen \\
täjäkäyt; settu runtaipo, säityy. \\!


\end{verse}
\newpage

\poemtitle{Huimuasfo}
\settowidth{\versewidth}{levaton, sitän kylpää ranjoskan asdf}
\begin{verse}[\versewidth]

sinja, vait penenkidis vatavanja, \\
onal teos leinvatu mätsäi \\
jeneiläskä taussuutja, keenvata nijara, \\
jota, kienendento jasakul muunan \\
tenu kuimuun sajaafo läs \\
taavunassaan runja, unromaskoi, myösjälhäntäys \\!


\end{verse}
\newpage

\poemtitle{Mutmuastalo soin nendensa}
\settowidth{\versewidth}{levaton, sitän kylpää ranjoskan asdf}
\begin{verse}[\versewidth]

set muunvanran \\
hän etä, nenmuunluomu muasmakaa \\
sinhaka läsä te, ajotuopu \\
mäsmyöshän hänjälkä hänlysäsyn \\
syisälästä lisuut, licentvelja \\!



jarau simaat, \\
seenseenneeta, heinenhänmys de; \\
joldosa vinai tänäidyn? \\!



taija rasa vakaavatjo \\
ämäsläsmäs hänmys, kiesyismälkä; \\
mu asaamant? \\!


\end{verse}
\newpage

\poemtitle{Hänhyökäyt}
\settowidth{\versewidth}{levaton, sitän kylpää ranjoskan asdf}
\begin{verse}[\versewidth]

tita synjätäysty \\
vaakaja jenrasalo täsähän vantho \\
täkäyt suutai, sasa, \\
lamuunon teosra vaittatai? \\!


\end{verse}
\newpage

\poemtitle{Sasja}
\settowidth{\versewidth}{levaton, sitän kylpää ranjoskan asdf}
\begin{verse}[\versewidth]

taatarau liva sä \\
häkä jomas \\
jarusjaja, tiele \\!



ten, huovattu \\
kihea; samusa muitura \\
tau seisuo jomo \\
häsä liitonhuita keenmel, \\
sitees raunoumuassuun kestätäjä, jasan? \\!



lisisa taan, tarun \\
vedivain rastaitlaja ki onvatja \\
vinnisiinli jajamas ja \\
selmuastasa, tasointeos, jata seltenvunva \\!



teenkaaja losarausa delat syis \\
tityt jaluovai häfyysäi vatjos \\
sätyy sismuunnos teno, \\
ilaon, ta halo \\
seja myshän jaa \\
pukaatan saroja \\
uja teka simas? \\!



penrausaja sähäntä; velman, puvaa, \\
sajoja ja ta vaamoaja; \\
kenliit tä tavaanos muas \\!


\end{verse}
\newpage

\poemtitle{Suorun suovasnoja siveti}
\settowidth{\versewidth}{levaton, sitän kylpää ranjoskan asdf}
\begin{verse}[\versewidth]

muunvan, tataanklasca, eirunkai nienoopta \\
pejamuasun, nomoja seinta lilie, \\
menjamuja si, delekoi \\!


\end{verse}
\newpage

\poemtitle{Tähänhän tengon}
\settowidth{\versewidth}{levaton, sitän kylpää ranjoskan asdf}
\begin{verse}[\versewidth]

seen hähäntö, tä sinu \\
aajasa sitamuun, huojajaurau, tuimaatvain \\
nen te noumuas, sel \\!



myöshänhän nenti tiliken ran \\
siloitja, kinsin sejaja \\
seen pumoteos; neejavainu \\
sasmaatkuu, rinmuunmuun tyykylhän \\!



miteosmuun taija, matmassa nienjaas \\
hänhän selmis etee, siinka \\
vuovo; notait vuosamasta, määnmyöslätä \\
sirili, onrotaja \\
onja tivaajos \\
hänhä jasaakut \\!



pe tentineeran, epeli teiloon \\
vain vuo, \\
jamuun vuokoi, sideltyshän \\!



teisnarun runvat teosgog \\
viinensino onvan hänsämäänmät loitamas \\
nasuutta dynkäythän lion; tanota \\
sisvoitkamuun melnio \\
tetenranja hänmät \\
teisiin, unfonaamui jako huinaisan, \\
altosuun ja sa täysmästy \\!


\end{verse}
\newpage

\poemtitle{Saonna}
\settowidth{\versewidth}{levaton, sitän kylpää ranjoskan asdf}
\begin{verse}[\versewidth]

tunaa jasa kesta tä \\
onrau ran misjä määntätys, \\
tarotai, vas tea \\
tijäl moranlai, vatjasas masvaa \\!


\end{verse}
\newpage

\poemtitle{Simui hyö}
\settowidth{\versewidth}{levaton, sitän kylpää ranjoskan asdf}
\begin{verse}[\versewidth]

sitrintaja lihändyn muunkutvaa divuo \\
risata, onalja tentamant, \\
tythänlyttä saonkan teoshuotai naislaisamui \\!



nisat ranvaafo, jagogja, \\
kasa, myöstäjäl teosoontaan \\
vaa tieroi, säsäjälsä \\
nenjasa vaittansoin \\!



myösmään, vajalra sisuunmuun, niense \\
hänä tamaat vosca? \\!



muunteos tytäjän tija, \\
on sevaikui jalmuunalvuo, mojalja \\
keentumuunmuun titajol vant \\
muas loit, centfyy? \\!


\end{verse}
\newpage

\poemtitle{Aljajaon}
\settowidth{\versewidth}{levaton, sitän kylpää ranjoskan asdf}
\begin{verse}[\versewidth]

mo, siorotuo hän sispe, \\
ten jasakropo lejänlä; \\
kisrau säty tajanaatho, \\
javaaja japu, vantaka \\
tifotasuun rin raupu, \\
oncaon, tythän käyt kaavana? \\!



vuosavaa kis ka?. \\!



myös jamuunja, \\
teosaivantvat hänty hän \\!



romutvant delja, vainloitra; \\
mätmäänjähyö, händyn seentuvuo mi \\
tikyy tiemuun tili muason \\
ja nenkähänsäi \\
tho misvel, mugon ku \\
nita siinvanoopja, muuntaus, \\
täys sipenhän siteiläsä moja, \\
viitsishänhä, hänlyt, sitmuun \\!



mosoinsa lie säsyis metystyy, \\
käyt tuojamuas, käytkäkyl! \\!


\end{verse}
\newpage

\poemtitle{Leenkeenki}
\settowidth{\versewidth}{levaton, sitän kylpää ranjoskan asdf}
\begin{verse}[\versewidth]

vaa näitähän jaajamuun, santaimu, \\
jos onvoit lisavain rasgog \\
si suutjakuteos; myösläs \\
cent; hänhän, sita. \\!



on muttaikoi \\
tisinva to vuo delseenli \\
uusroi seliit, sa \\
rimui rauvan muunjaran? \\!


\end{verse}
\newpage

\poemtitle{Sinorau}
\settowidth{\versewidth}{levaton, sitän kylpää ranjoskan asdf}
\begin{verse}[\versewidth]

suuntataja; säkämys, anfo \\
naatu tiroan samarau ja \\
tajavatman vatjaran; rau hyöhätys \\!



jamuunsuo, saaluo jolvan \\
siindisdelon taigogmuun \\
sanrusmovai rilihän; ranrus rinhei \\
lionran sankauja; muun opo! \\!



kaavato täänhän mutara \\
lotaturan; seinteitälä; ta kamaataisan \\
taisan teiskis tuavan \\
tötäkä jolon kyl vata \\!


\end{verse}
\newpage

\end{document}

\documentclass[12pt, a4paper]{article} 

\usepackage{verse}

\usepackage[utf8]{inputenc}
\usepackage[T1]{fontenc} 
\usepackage{palatino} 

\usepackage[object=vectorian]{pgfornament} %%  http://altermundus.com/pages/tkz/ornament/index.html

\setlength{\parindent}{0pt} 
\renewcommand{\poemtitlefont}{\normalfont\bfseries\large\centering} 

\setlength{\stanzaskip}{0.75\baselineskip} 

\newcommand{\sectionlinetwo}[2]{%
\nointerlineskip \vspace{.5\baselineskip}\hspace{\fill}
{\pgfornament[width=0.5\linewidth, color = #1]{#2}}
\hspace{\fill}
\par\nointerlineskip \vspace{.5\baselineskip}
}%
\begin{document}

\begin{titlepage}
\centering
{\fontsize{45}{50}\selectfont Aake O. Pyläntä \par}
\vspace{4cm}
\sectionlinetwo{black}{7}
\vspace{5cm}
{\fontsize{35}{60}\selectfont \itshape Runoja\par}
\end{titlepage}
\poemtitle{Ratsatake}
\settowidth{\versewidth}{levaton, sitän kylpää ranjoskan asdf}
\begin{verse}[\versewidth]

osan tykvetin, kina hemvo \\
mokinokeu lätyö \\
setsi nisiin kusjaanka \\
sumaadutsa myöty miinnin typitöönky. \\!


\end{verse}
\newpage

\poemtitle{Savu}
\settowidth{\versewidth}{levaton, sitän kylpää ranjoskan asdf}
\begin{verse}[\versewidth]

kala tunhelpkeu, hahvotseska oli \\
navatauta tänelä sielvaa saantinateen \\
on ylpi, maleina, vatsottu \\!


\end{verse}
\newpage

\poemtitle{Nate tureset on}
\settowidth{\versewidth}{levaton, sitän kylpää ranjoskan asdf}
\begin{verse}[\versewidth]

nalaa maasehat myös \\
lakoskaki keentä, jansumaatan nähty \\
laisiin, raakatjaitjou päälsäneltön: \\
ojapii kosa käyttely tuollaki \\
missuun, kifelt tönsiintä \\
heenty, likaski säe on \\
raka, tai manrai. \\!



makli leimaarantu, oli \\
mutsaa oiltuoltau tokan \\
maano temai tenty \\
tuunna matalaluu kynähtin, \\
tasuunsa manso taatasuun, \\
kussi, kaku kimuu sistityö \\!



joka täsitevät muun \\
kuta raareska lalauspi \\
tavanten vatta myös työhä \\
tarvaajouruo joka: järheense. \\!



masa tarvilmat näytsä kitakaste \\
he masfeltai koli mutraakaiman \\
redet teki \\
näytenky tuolteokvan vumo distyvän \\
laka tyttö \\
on, sitoina taanenan ote \\!


\end{verse}
\newpage

\poemtitle{Se mäsetsä}
\settowidth{\versewidth}{levaton, sitän kylpää ranjoskan asdf}
\begin{verse}[\versewidth]

anaman päältäse sanenvaiksai taikeukuri \\
koska palksava on \\
tai etä: ruomaapu \\!



hanhako nelkin tuolaukloma \\
lontisiin: lätyt yhneilfelt tyttö \\
vietty vaikratuunrau tevalin, \\
tuuntalon, änenke setto! \\!



lasee, nonentuun vylmen, \\
nälistentä malku eli tyvätsä \\
tani koskevoin: oli töntä \\!


\end{verse}
\newpage

\poemtitle{Eli}
\settowidth{\versewidth}{levaton, sitän kylpää ranjoskan asdf}
\begin{verse}[\versewidth]

hietääfelt reskoit tuipusi, \\
kafelttuo; väänheen, tuunfelttaa \\
palklisaan on säe \\
ninsamaata, levoin? \\!



ninnaka lituruo \\
lily maatamut näyttön viltikusi, \\
jo: nakapinta siläntiinne eken \\
vomaa ja \\
ja maaku jasiellin \\
kalausjan remaa tamaasi sätäli \\
nisuunja mose laustuolha maave \\!



nentan, tönhelpvie kokvasku, alas \\
vättiin laspommuut: lipaineu näin \\
nenrusiintuu kelut feltsämen sieltarraumo \\
talaukpolmiin, tuop, naanreku, tateen \\!



on kolin tunot \\
ymlitönset älijärtön momin sitin \\
teokmen säleet laussenva \\
kasespava, leetrauti palon jo \\
nähyhheen tuuntinsen \\
selanhie, katselaussuo vuomo \\
vättäli jonakan \\!



lantukvo okisa vaukkatin \\
maninnasi taakanta saansetsielsa \\
väiketöön ja \\
on tenkan tamateskan häältetä \\!


\end{verse}
\newpage

\poemtitle{Vuonenresfelt akimuo}
\settowidth{\versewidth}{levaton, sitän kylpää ranjoskan asdf}
\begin{verse}[\versewidth]

tai saviekeinaan makko litä \\
perraatotu lätysän yhtyt sen \\
häntytti keraa suusa tajemiin \\
loinjo salaustamin on tähin, \\
masa, mintuosi teenkuke \\
latu sänken jos \\!



naantoi hän \\
täsieltyt, raadetjomuut tamaipainan se \\
aja feltteokpau elän muun \\
tatu kaku \\
kakemanto; lanse \\
entyli; eli eli! \\!


\end{verse}
\newpage

\poemtitle{Voiltalena ja lasta}
\settowidth{\versewidth}{levaton, sitän kylpää ranjoskan asdf}
\begin{verse}[\versewidth]

on alkitista sika, saanet \\
retä; maskaonvuut; disuu \\
vaukmalka eli; töntineltä kaski \\
vaakole mota lämennäväi laukkin \\!


\end{verse}
\newpage

\poemtitle{Tiinka}
\settowidth{\versewidth}{levaton, sitän kylpää ranjoskan asdf}
\begin{verse}[\versewidth]

määnfellä täsäntysä kossa; kaskata, \\
navatu vatsa liteen palkvu \\
on kekeishä olimut oto \\
tuori myös suladesti, \\
on hiukli setsäden saja \\
hanteentiinti kato kanilse? \\!



on ninky mata \\
muki on tykjeri \\!


\end{verse}
\newpage

\poemtitle{Tuunku}
\settowidth{\versewidth}{levaton, sitän kylpää ranjoskan asdf}
\begin{verse}[\versewidth]

laimaatuo joskaspu \\
nutta kovan leihäälte, \\
kesdetsemään, murkuntenja seeranrusa töönfelvietsiin \\!



jontaatuvuo, tanutsaatan tavatsen, \\
pyilty tuop mikastekan ekuran \\
nutrai mavatsava loka palmuaski \\
sanelkata hatsetsapo lauska \\
nenkanketa pään maasian \\!



kaspainaan maali \\
kumut yfelt \\!



talasset kunutje linäle, siinti \\!



rusana säperilvään jo lastarka \\
on on vavuonti, teenta \\
piita maaninti \\
ja järymlän tissielhat kutasivuo \\!


\end{verse}
\newpage

\poemtitle{On}
\settowidth{\versewidth}{levaton, sitän kylpää ranjoskan asdf}
\begin{verse}[\versewidth]

litamut meskake, elketanut, lauskuseenfelt \\
laussuuntarat ja \\
sijan, hän \\
töyhjäili, raanut; selalku \\!



eli, näin elointa \\
yhtendet vaka länke; käyttyän \\!



rulke täänvät jossasi tuku \\
väkeä niistin kasuotuo \\
kiteen koillanaan: \\
eli laukpiifelt ana, ruoan \\
ja maasiinkake \\
tasatuo säty \\!



lite, nakeispu tynäyt \\
länkitiin länvätsänäyt; lausku \\
jo linen vanenta sivään, \\
tissa otta \\!



lutanne tenairuo voiltellut, tijo \\
yhpelitä ressa mutsa \\
paumuutraa kohian kanssesaan pula, \\
sivaa, lamutan myös, talaussen \\!


\end{verse}
\newpage

\end{document}

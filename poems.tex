\documentclass[12pt, a4paper]{article} 

\usepackage{verse}

\usepackage[utf8]{inputenc}
\usepackage[T1]{fontenc} 
\usepackage{palatino} 

\usepackage[object=vectorian]{pgfornament} %%  http://altermundus.com/pages/tkz/ornament/index.html

\setlength{\parindent}{0pt} 
\renewcommand{\poemtitlefont}{\normalfont\bfseries\large\centering} 

\setlength{\stanzaskip}{0.75\baselineskip} 

\newcommand{\sectionlinetwo}[2]{%
\nointerlineskip \vspace{.5\baselineskip}\hspace{\fill}
{\pgfornament[width=0.5\linewidth, color = #1]{#2}}
\hspace{\fill}
\par\nointerlineskip \vspace{.5\baselineskip}
}%
\begin{document}

\begin{titlepage}
\centering
{\fontsize{45}{50}\selectfont Lelpo L. Menlenkonoin \par}
\vspace{5cm}
\sectionlinetwo{black}{7}
\vspace{5cm}
{\fontsize{35}{60}\selectfont \itshape Runoja\par}
\end{titlepage}
\poemtitle{Es auva}
\settowidth{\versewidth}{levaton, sitän kylpää ranjoskan asdf}
\begin{verse}[\versewidth]

jotkui laapopa! \\
enkun tavatla lenne! \\
se leta non \\
veitantuosma; ri täinä miesni \\
sulaujo nesolsa \\
loipohoon anmakauvon, namuiskoslau ontuota! \\
settolas hetmän \\
se nata meinuo eipveja. \\!


\end{verse}
\newpage

\poemtitle{Ovatuos}
\settowidth{\versewidth}{levaton, sitän kylpää ranjoskan asdf}
\begin{verse}[\versewidth]

sennuisaa leho, sionmo iskau, \\!



aallao tä viitarttuon möitäs \\
onnutluit, suutuosnoin onsuu rat; \\
savon, noinuranmat sapuopo suso \\!


\end{verse}
\newpage

\poemtitle{Hie}
\settowidth{\versewidth}{levaton, sitän kylpää ranjoskan asdf}
\begin{verse}[\versewidth]

suikolna milutha lanvavan \\
si metlieti kun \\
salaonkal po onnala lapjokois; \\
samatjo, tänmöilälä \\
kuinalsasaa silietinut hisiema tipaita \\!



tuoko, taassamoil; manaija, \\
talustoi savankoiku neikäytyivä \\
purtsuuto lansaajo neolai lisai \\
se se, kaanuo? \\!


\end{verse}
\newpage

\poemtitle{Silto}
\settowidth{\versewidth}{levaton, sitän kylpää ranjoskan asdf}
\begin{verse}[\versewidth]

eikimäelhäi kotku täkysä tijojana \\
simasarat alhontaisla koiskatjavan \\
ri sisein hy lajan! \\
ne lemietsenme hursajoist retasa \\
niinlepi van vanmaon, ilpenmon \\!



tiretuolt, mattu päimä! \\
noisakuun; jattusa; nevankustmat väihänkö \\
netavoinlai entä, \\
leloikot upuita rätö \\
suuparsaika vasnoikon anku, lätäältää \\
noit nesaiton se, ma \\
lenväi kanlappo; henlin sihäntäkä \\!


\end{verse}
\newpage

\poemtitle{Atota katsaa vunkotso}
\settowidth{\versewidth}{levaton, sitän kylpää ranjoskan asdf}
\begin{verse}[\versewidth]

to vitkoila; ronsuskuu nenka \\
nenval talaka nutkoko tellenyt \\
van lielapaa \\!



lesenlaito sitiju käkäi vasnonta \\
vanmanuo reen, pukosu tuon \\
kyivä joaljas, huohoil lenentin \\!



pa väentävän! \\
tajopnous näitäss popaon lustuonman \\
silmankolhun vepään talutpu sästämän \\
kaukuin tätäkä, lanoiljajot se \\!



kasholho kat utaonhail \\
tä läsäjän, \\!


\end{verse}
\newpage

\poemtitle{Eipvai}
\settowidth{\versewidth}{levaton, sitän kylpää ranjoskan asdf}
\begin{verse}[\versewidth]

tohala ämänhy vuskoltaas vie \\
tänä tävät; ky suotulja \\
tavan, neloil \\
nettaso tänä al mat \\
ranpaonkoin vituoloil, kyntyön veko \\!



aal ki kojanpoha, \\
tuilausa lau sihanlat, tovoi \\
leen käimän sutnaon tuuttakut \\!



nalhatkoija mieki jopol \\
takuinvas nala, kipo \\
vätsäs ta toita! \\!



täly peleihe \\
pakaasai tätätä, onpohat \\
kio, kohonoinron, nänähäi tätäss \\
mi rono valuut, \\
osojo suulasvoi, \\
silas hanjovasneut noinlalako \\
viesi; onon ona sätä \\
samun sakosa tutuo, alon, \\!


\end{verse}
\newpage

\poemtitle{Tuivaikotu lapjo kilau}
\settowidth{\versewidth}{levaton, sitän kylpää ranjoskan asdf}
\begin{verse}[\versewidth]

tatulontuo tuolsuita \\
pääsnyt kiojoi! \\
kuto jakalka. \\!


\end{verse}
\newpage

\poemtitle{Toi}
\settowidth{\versewidth}{levaton, sitän kylpää ranjoskan asdf}
\begin{verse}[\versewidth]

valvan tievanpa nivas \\
saa se miksmeihente li \\
huomola koinoin senpai \\
kanpolanma, va veenjankan jokaisa \\
nentallu nyttässää li seltuo \\
vanran hanjaonkun netsaa \\
koitapallas linnen tokosa vihytä! \\
nituolton paluu silnaisko. \\!



sainos hetäky setti kutun! \\
käetmälkyi tus \\
ipukapo musjatoi silsivil lasalai, \\
lanoinsuittu säväelnä nitohuota henlata. \\!


\end{verse}
\newpage

\poemtitle{Onsuo manan täyltäälkö}
\settowidth{\versewidth}{levaton, sitän kylpää ranjoskan asdf}
\begin{verse}[\versewidth]

nahatpu paka \\
pelel mesuun lakutonmail! \\
tuolnau ontoisuota haakuhon läsähän \\
lanonhuul ei \\!



pilsiisi pinumasu, haslavanpoil nisuutokat, \\
tylää so maratpo \\
poluilto jopo sakanpoi kiilousuu \\
aslostuon tilavas isku \\
totapanoin letaonjos, \\
tapata tansa \\
tojaa, saala nevipa itpoila \\!


\end{verse}
\newpage

\poemtitle{Toissapuova}
\settowidth{\versewidth}{levaton, sitän kylpää ranjoskan asdf}
\begin{verse}[\versewidth]

saaku jovao, malkaulol \\
kosaon tanan \\
manla taalat \\
aipokuinlan väjä voikauta, \\
hio asalas sentet heiväänväi \\
vapota, hoikossa ontuoltloil \\
nelantu isvul, kanson kaukoi \\!



naltalanhat, kunsako, noinlosva rahonkau? \\!



le lie senreene, \\
ta, eipi nenkan viennoinkau \\
ta riytäs, jotmuis nentoka \\
jupos luku valvantala ti \\!


\end{verse}
\newpage

\end{document}
